\documentclass{article}
\usepackage[a4paper]{geometry}
\usepackage{amsmath,amsthm,amssymb}
\usepackage[framemethod=tikz]{mdframed}
\usepackage{etoolbox}
\usepackage[utf8]{inputenc}
\usepackage[ngerman]{babel}

\title{Skript Logik- und Modelltheorie}
\author{Felix Linker}
\date{SS2018}

\theoremstyle{definition}
\newtheorem{dfn}{Definition}[section]
\newtheorem{konstr}[dfn]{Konstruktion}
\newtheorem{lem}[dfn]{Lemma}
\newtheorem{satz}[dfn]{Satz}

\theoremstyle{plain}
\newtheorem*{anm}{Anmerkung}
\newtheorem*{bsp}{Beispiel}
\newtheorem{bem}[dfn]{Bemerkung}

\mdfdefinestyle{standard}{
    skipabove=5pt,
    skipbelow=5pt,
    innerbottommargin=12pt,
    leftmargin=5pt,
    rightmargin=5pt,
    linewidth=1pt,
    roundcorner=4pt
}
\surroundwithmdframed[style=standard]{dfn}
\surroundwithmdframed[style=standard]{konstr}
\surroundwithmdframed[style=standard]{lem}
\surroundwithmdframed[style=standard]{satz}

\newcommand{\calA}{\mathcal{A}}
\newcommand{\calC}{\mathcal{C}}
\newcommand{\calF}{\mathcal{F}}
\newcommand{\calG}{\mathcal{G}}
\newcommand{\calK}{\mathcal{K}}
\newcommand{\calM}{\mathcal{M}}
\newcommand{\calN}{\mathcal{N}}
\newcommand{\calR}{\mathcal{R}}

\newcommand{\sign}[1]{(\calC_{#1}, \calF_{#1}, \calR_{#1}, \sigma'_{#1})}
\newcommand{\struc}[3]{\big(#1, (c^{#2})_{c \in \calC_{#3}}, (f^{#2})_{f \in \calF_{#3}}, (R^{#2})_{R \in \calR_{#3}}\big)}

\sloppy

\begin{document}

    \maketitle

    \section*{Grundlagen}

    Mathematische Objekte bestehen aus Grundmengen ggf. Relationen, Funktionen und Konstanten.

    Einfache Aussagen betreffen nur Elemente der Grundmenge und haben keine unendlichen Dis- oder Konjunktionen.
    Dies sind prädikatenlogische Aussagen erster Stufe.

    \textit{Monadische Aussagen 2. Stufe erlaubten Quantifizierung über Teilmengen der Grundmenge; Aussagen 2. Stufe erlaubten zstl. Quantifizierung über Funktionen und Relationen.}

    \section{Strukturen}

    \begin{dfn}
        Eine \underline{Signatur} $ \sigma $ ist ein Quadrupel
        \begin{equation}
            \sigma = \sign{}
        \end{equation}
        mit einer Menge von Konstantensymbolen $ \calC $, einer Menge von Funktionssymbolen $ \calF $, einer Menge von Relationensymbolen $ \calR $, einer Stelligkeitsfunktion $ \sigma' : \calF \cup \calR \rightarrow \mathbb{N} $.
    \end{dfn}

    \begin{dfn}
        Eine \underline{Struktur} $ \calM $ ist ein Quadrupel
        \begin{equation}
            \calM = \struc{M}{\calM}{}
        \end{equation}
        mit einer Menge $ M $ (oftmals $ M \neq \emptyset $), Indexmengen $ \calC, \calF, \calR $.
        Wobei gilt $ c^\calM \in M $ für $ c \in \calC $, $ f^\calM : M^{n_f} \rightarrow M $, $ n_f \in \mathbb{N} $ für $ f \in \calF $, $ R^\calM \subseteq M^{m_R} $ für $ R \in \calR $.
    \end{dfn}

    $ \calM $ heißt \underline{$ \sigma $-Struktur} bzw. $ \calM $ und $ \sigma $ \underline{passen zueinander}, falls:
    \begin{enumerate}
        \item $ n_f = \sigma'(f) $ für $ f \in \calF $
        \item $ m_R = \sigma'(R) $ für $ R \in \calR $
    \end{enumerate}

    $ c^\calM $ heißt auch \underline{Interpretation} von $ c $ in $ \calM $; Analoges gilt für $ f^\calM, R^\calM $.

    \begin{dfn}
        Seien $ \calM = \struc{M}{\calM}{} $, $ \calN = \struc{N}{\calN}{} $ zwei $ \sigma $-Strukturen.
        Ein Homomorphismus von $ \calM $ nach $ \calN $ ist eine Abbilung $ h : M \rightarrow N $ mit:
        \begin{enumerate}
            \item $ h(c^\calM) = c^\calN $ für allen Konstantensymbole $ c \in \calC $
            \item $ h\big(f^\calM(a_1, ..., a_n)\big) = f^\calN\big(h(a_1), ..., h(a_n)\big) $ für alle Funktionssymbole $ f \in \calF $ mit $ n = \sigma'(f) $, $ a_1, ..., a_n \in M $.
            \item \label{itm:homomorphismus-3} $ (a_1, ..., a_n) \in R^\calM \Rightarrow \big(h(a_1), ..., h(a_n)) \in R^\calN $ für alle Relationensymbole $ R \in \calR $, $ n = \sigma'(R) $, $ a_1, ..., a_n in M $.
        \end{enumerate}
    \end{dfn}

    Für einen Homomorphismus $ h $ von $ \calM $ nach $ \calN $ schreibt man auch $ h : \calM \rightarrow \calN $.

    $ h $ heißt \underline{stark}, wenn für alle $ R \in \calR $ mit $ n = \sigma'(R) $, $ b_1, ..., b_n \in h(M) \subseteq N $ mit $ (b_1, ..., b_n) \in R^\calN $ gilt:
    \begin{equation}
        \label{eq:staerke}
        \exists a_1, ..., a_n \in M : (a_1, ..., a_n) \in R^\calM, h(a_i) = b_i, 1 \leq i \leq n
    \end{equation}

    $ h $ heißt \underline{Einbettung}, falls $ h $ stark und injektiv ist.
    Ist $ h $ injektiv, dann lassen sich Bedingung \ref{itm:homomorphismus-3} und Gleichung \eqref{eq:staerke} folgendermaßen zusammenfassen; für alle $ R \in \calR $, $ n = \sigma'(R) $, $ a_1, ..., a_n \in M $:
    \begin{equation}
        (a_1, ..., a_n) \in R^\calM \Leftrightarrow \big(h(a_1), ..., h(a_n)\big) \in R^\calN
    \end{equation}

    $ h $ heißt Isomorphimus, wenn $ h $ eine surjektive Einbettung ist.
    Man schreibt $ \calM \cong \calN $, wenn ein Isomorphimus $ h : \calM \rightarrow \calN $ existiert.
    Sind $ h : \calM \rightarrow \calN $, $ g : \calN \rightarrow \calK $ Isomorphismen, dann sind auch folgende Funktionen ein Isomorphimus:
    \begin{itemize}
        \item $ h^{-1} $
        \item $ g \circ h : \calM \rightarrow \calK $
    \end{itemize}

    $ h : \calM \Rightarrow \calM $ heißt \underline{Automorphismus von $ \calM $}, wenn $ h $ ein Isomorphimus ist.
    Die Struktur $ Aut(\calM) = \big( \{ h : \calM \rightarrow \calM \mid h \text{ Automorphismus} \}, \circ \big) $ mit Komposition $ \circ $ ist eine Gruppe.

    \begin{dfn}
        Seien $ \calM = \struc{M}{\calN}{} $, $ \calN = \struc{N}{\calN}{} $ $\sigma $-Strukturen.
        $ \calN $ heißt \underline{Teil-/Unter-/Substruktur} von $ \calM $, falls:
        \begin{enumerate}
            \item $ N \subseteq M $
            \item $ c^\calN = c^\calM $ für $ c \in \calC $
            \item $ f^\calN(\bar{a}) = f^\calM(\bar{a}) $ für $ f \in \calF $, $ \bar{a} \in N^{\sigma'(f)} $
            \item $ \bar{a} \in R^\calN \Leftrightarrow \bar{a} \in R^\calM $, d.~h. $ R^\calN = R^\calM \cap N^{\sigma'(R)} $ für $ \bar{a} \in N^{\sigma'(R)} $
        \end{enumerate}
    \end{dfn}

    Wenn $ \calN $ eine Teilstruktur von $ \calM $ ist, schreibt man auch $ \calN \subseteq \calM $.
    $ \calM $ wird dann auch \underline{Ober- oder Erweiterungsstruktur von $ \calN $} genannt.

    \begin{dfn}
        Seien $ \sigma_0 = \sign{0} $, $ \sigma_1 = \sign{1} $ zwei Signaturen.
        Wir definieren $ \sigma_0 \subseteq \sigma_1 $, d.~h. $ \calC_0 \subseteq \calC_1 $, $ \calF_0 \subseteq \calF_1 $, $ \calR_0 \subseteq \calR_1 $ und $ \sigma'_0 = \sigma'_1 \vert_{\calF_0 \cup \calR_0} $ .

        Sei $ \calM = \struc{M}{\calM}{1} $ eine $ \sigma_1 $-Struktur.
        Dann heißt $ \calM \vert_{\sigma_0} = \struc{M}{\calM}{0} $ das \underline{Redukt} von $ \calM $ auf $ \sigma_0 $ od. auch $ \sigma_0 $-Redukt.
        Umgekehrt heißt $ \calM $ \underline{Expansion} von $ \calN := \calM \vert_{\sigma_0} $ auf $ \sigma_1 $, d.~h. $ \calM $ entsteht aus $ \calN $ durch Hinzunahme geeigneter (bzgl. $ \sigma_1 $) Konstanten/Funktionen/Relationen.
    \end{dfn}

    \begin{bsp}
        Betrachten wir die Gruppe $ \calG = (G, 1^\calG, \circ^\calG) $.
        $ \calG $ ist ein Redukt von $ (G, 1^\calG, \circ^\calG, \circ^{-1\calG}) $.
        Von $ \calR = (R, 0, 1, +, \cdot) $ ist $ (R, 0, +) $ ein Redukt.
    \end{bsp}

    Eine Signatur $ \sigma = \sign{} $ heißt \underline{konstantenlos}, falls $ \calC = \emptyset $ und \underline{(rein) relational}, falls $ \calC = \calF = \emptyset $.

    \section{Sprachen und Formeln}

    Wunsch: Formeln der Form $ \forall x. (x \geq 0 \rightarrow \exists y. x = y \cdot y) $.

    Gegeben eine Signatur $ \sigma = \sign{} $.
    \underline{Grundsymbole} der Sprache $ L = L(\sigma) $ sind:
    \begin{enumerate}
        \item abzählbar viele Variablen $ v_n $, $ n \in \mathbb{N} $
        \item die Symbole von $ \sigma $ auf $ \calC \cup \calF \cup \calR $ (nichtlogische Symbole)
        \item Logisch Zeichen: $ \neg, \land, = $
        \item Quantor $ \exists $
        \item Klammersymbole $ ( $ und $ ) $
    \end{enumerate}

    Die Menge der \underline{Terme} von $ \sigma $ oder $ L(\sigma) $ ist die kleinste Menge für die gilt:
    \begin{enumerate}
        \item Alle Variablen $ v_n $, $ n \in \mathbb{N} $ sind ein Term.
        \item Alle Konstantensymbole $ c \in \calC $ sind ein Term.
        \item Wenn $ t_1, ..., t_n $ Terme sind, $ f \in \calF $ mit $ \sigma'(f) = n $, dann ist auch $ f(t_1, ..., t_n) $ ein Term.
    \end{enumerate}

    Die Menge der \underline{Atomformeln} bzgl. $ \sigma $ ist die kleinste Menge für die gilt:
    \begin{enumerate}
        \item Wenn $ t_1, t_2 $ Terme sind, dann ist $ t_1 = t_2 $ eine Atomformel,
        \item Wenn $ t_1, ..., t_n $ Terme sind, $ R \in \calR $ mit $ \sigma'(R) = n $, dann ist $ R(t_1, ..., t_n) $ eine Atomformel.
    \end{enumerate}

    Die Menge der \underline{Formeln} bzgl. $\sigma $ ist die kleineste Menge für die gilt:
    \begin{enumerate}
        \item Jede Atomformel ist eine Formel.
        \item Wenn $ \varphi, \psi $ Formeln sind, $ x = v_n $ $ n \in \mathbb{N} $ Variable, dann sind auch $ \neg \varphi$, $ (\varphi \land \psi) $, $ (\exists x) \varphi $ Formeln.
    \end{enumerate}

    $ L(\sigma) := $ Menge der Formeln bzgl. $ \sigma $.
    Abkürzungen für zwei Formeln $ \varphi , \psi \in L(\sigma) $ sind wie üblich definiert:
    \begin{itemize}
        \item $ \varphi \lor \psi := \neg (\neg \varphi \land \neg \psi) $
        \item $ \varphi \rightarrow \psi := \neg \varphi \lor \psi $
        \item $ \varphi \leftrightarrow \psi := (\varphi \rightarrow \psi) \land (\psi \rightarrow \varphi) $
        \item $ (\forall x) \varphi := \neg (\exists x) \neg \varphi $
    \end{itemize}

    Die Grundsymbole sind von $ L(\sigma) $ sind absichtlich so spärlich definiert, da Beweise oftmals über den Aufbau von Formeln geführt werden und in diesem Fall minimal-viele Zeichen in Beweisen berücksichtigt werden müssen.

    Seien $ x $ eine Variable $ \varphi $ und eine Formel. Der \underline{(Wirkungs-)Bereich} eines Quantors $ (\exists x) \varphi $ ist definiert als $ \varphi $.
    Ein Vorkommen von $ x $ im Bereich von $ (\exists x) $ heißt \underline{gebunden}.
    Ist $ x $ nicht im Bereich eines Quantors, so heißt dieses Vorkommen \underline{frei}.
    $ x $ ist eine \underline{freie Variable}, wenn $ x $ an mindestens einer Stelle frei vorkommt; ansonsten heißt $ x $ \underline{gebunden}.

    Eine Formel $ \varphi $ heißt \underline{Aussage} oder \underline{Satz}, falls $ \varphi $ keine freien Variablen enthält.

    Nun sei $ \calM = \struc{M}{\calM}{} $ eine $ \sigma $-Struktur.
    Wir wollen die Gültigkeit von Formeln $ \varphi $ in $ \calM $ induktiv definieren.
    Jedes Tupel $ x = (x_i)_{i \in \mathbb{N}} $ mit $ x_i \in M $ ($ x \in M^{\mathbb{N}} $) heißt eine Bewertung der Variablen $ v_i $, $ i \in \mathbb{N} $ von $ L $.
    Ist $ n \in \mathbb{N} $ und $ a \in M $, so sei $ x(n/a) := (x_1, ..., x_{n-1}, a, x_{n+1}, ...) $ die \underline{Aktualisierung} od. \underline{Update} von $ x $ an der Stelle $ n $ durch $ a $.

    \begin{dfn}
        Sei $ \calM = \struc{M}{\calM}{} $ eine $ \sigma $-Struktur, $ x \in M^{\mathbb{N}} $.
        Der \underline{Wert} eines Terms $ t $ in $ (\calM, x) $ sei:
        \begin{enumerate}
            \item $ t^{(\calM, x)} := x_n $, falls $ t = v_n $ Variable
            \item $ t^{(\calM, x)} := c^\calM $, falls $ t = c $ Konstantensymbol
            \item $ t^{(\calM, x)} := f^\calM(t_1^{(\calM, x)}, ..., t_n^{(\calM, x)}) $, falls $ t = f(t_1, ..., t_n) $ mit Funktionssymbol $ f \in \calF $, $ \sigma'(f) = n $ und Termen $ t_1, ..., t_n $
        \end{enumerate}
    \end{dfn}

    \begin{dfn}
        Wir definieren: $ (\calM, x) $ \underline{erfüllt} die Formel $ \varphi $ oder $ \varphi $ \underline{gilt} in $ (\calM, x) $ (wir schreiben $ (\calM, x) \models \varphi $ od. $ \calM \models \varphi[x] $), wie folgt:
        \begin{itemize}
            \item $ (\calM, x) \models t_1 = t_2 $, wenn $ t^{(\calM, x)}_1 = t^{(\calM, x)}_2 $
            \item $ (\calM, x) \models R(t_1, ..., t_n) $, wenn $ (t_1^{(\calM, x)}, ..., t_n^{(\calM, x)}) \in R^\calM $
            \item $ (\calM, x) \models \varphi \land \psi $, wenn $ (\calM, x) \models \varphi $ und $ (\calM, x) \models \psi $
            \item $ (\calM, x) \models \neg \varphi $, wenn nicht $ (\calM, x) \models \varphi $
            \item $ (\calM, x) \models (\exists v_n) \varphi $, wenn ein $ a \in M $ mit $ (\calM, x(n/a)) \models \varphi $ existiert
        \end{itemize}
    \end{dfn}

    \begin{anm}
        Sei $ \calM = \struc{M}{\calM}{} $ eine $ \sigma $-Struktur, $ x \in M^{\mathbb{N}} $ eine Bewertung, $\varphi, \psi \in L $.
        \begin{enumerate}
            \item $ (\calM, x) \models \varphi \lor \psi \Leftrightarrow (\calM, x) \models \varphi $ oder $ (\calM, x) \models \psi $\\
            $ (\calM, x) \models \varphi \rightarrow \psi \Leftrightarrow $ wenn $ (\calM, x) \models \varphi $, dann $ ( \calM, x) \models \psi $
            \item $ (\calM, x) \models (\forall v_n) \varphi \Leftrightarrow $ für alle $ a \in M gilt: \big(\calM, x(n/a)\big) \models \varphi $
            \item Sei $ y \in M^{\mathbb{N}} $ so, dass für alle freien Variablen $ v_n $ in $ \varphi $ gilt: $ x_n = y_n $.
            Dann: $ (\calM, x) \models \varphi \Leftrightarrow (\calM, y ) \models \varphi $. \label{itm:freie-variablen-gleich}
        \end{enumerate}
    \end{anm}

    \begin{proof}
        ~\par
        \begin{enumerate}
            \item Übungsaufgabe
            \item \begin{align*}
                (\calM, x) \models (\forall v_n) \varphi & \Leftrightarrow (\calM, x) \models \neg (\exists v_n) \neg \varphi \\
                & \Leftrightarrow \text{nicht: } (\calM, x) \models (\exists v_n) \neg \varphi \\
                & \Leftrightarrow \text{nicht ex. } a \in M: \big(\calM, x(n/a)\big) \models \neg \varphi \\
                & \Leftrightarrow \text{nicht ex. } a \in M: \text{nicht: } \big(\calM, x(n/a)\big) \models \varphi \\
                & \Leftrightarrow \text{für alle } a \in M: \big(\calM, x(n/a)\big) \models \varphi
            \end{align*}
            \item Übungsaufgabe; Hinweis: Per Induktion über Formelaufbau.
            \begin{enumerate}
                \item $ \forall t: t^{(\calM, x)} = t^{(\calM, y)} $ \label{itm:aussage-1}
                \item Induktion: $ \varphi = (t_1 = t_2) $, $ \varphi = R(t_1, ..., t_n) $ mit \ref{itm:aussage-1}, $ \varphi = \psi \land \xi $, $ \varphi = \neg \psi $, $ \varphi = (\exists v_n) \psi $
            \end{enumerate}
        \end{enumerate}
    \end{proof}

    Wegen Bemerkung \ref{itm:freie-variablen-gleich} hängt die Gültigkeit von $ (\calM, x) \models \varphi $ \underline{nur} von den ersten $ x_1, ..., x_n $ ab, die alle freien Variablen $ v_i $ in $ \varphi $ überdecken.
    Daher: Wir schreiben $ \calM \models \varphi[x_1, ..., x_n] $ falls $ (\calM, x) \models \varphi $ und $ n \geq max\{i \mid v_i \text{ frei in } \varphi\} $

    Ist $ \varphi $ ein Satz, dann gilt:
    \begin{align*}
        \calM \models \varphi & :\Leftrightarrow \exists x \in M^{\mathbb{N}}: (\calM, x) \models \varphi \\
        & \Leftrightarrow \forall x \in M^\mathbb{N}: (\calM, x) \models \varphi \\
        & \Leftrightarrow: \varphi \text{ ist wahr in } \calM
    \end{align*}

    $ \varphi $ \underline{universell richtig} gdw. $ \forall \sigma$-Strukturen $ \calM: \calM \models \varphi $.
    \begin{bsp}
        $ (\forall x) x = x $ bzw. $ (\forall v_n) v_n = v_n $
    \end{bsp}

    Zwei $ L $-Strukturen $ \calM, \calN $ heißen \underline{elementar äquivalent}, falls
    \begin{align}
        & \forall \text{ Sätze } \sigma: \calM \models \sigma \Leftrightarrow \calN \models \sigma \label{eq:elem-equiv-1} \\
        \Leftrightarrow & \forall \text{ Sätze } \sigma: \calM \models \sigma \Rightarrow \calN \models \sigma \label{eq:elem-equiv-2}
    \end{align}

    \begin{proof}
        Es gelte \eqref{eq:elem-equiv-2}, wir zeigen \eqref{eq:elem-equiv-1}.
        Sei $ \sigma $ beliebiger Satz.
        $ \calM \models \sigma \Rightarrow \calN \models \sigma $
        gilt trivialerweise nach Vorbedingung \eqref{eq:elem-equiv-2}.

        Es bleibt zu zeigen, dass $ \calN \models \sigma \Rightarrow \calM \models \sigma $.
        Es gelte $ \calN \models \sigma $.
        Angenommen, es gilt nicht: $ \calM \models \sigma $, dann muss gelten, dass $ \calM \models \neg \sigma $.
        Offensichtlich ist $ \neg \sigma $ ein Satz.
        \eqref{eq:elem-equiv-2} $ \Rightarrow \calN \models \neg \sigma $, also nicht $ \calN \models \sigma $. Widerspruch.
    \end{proof}

    Sei $ \calK $ eine Klasse von $ L $-Strukturen einer festen Sprache $ L $.
    eine Eigenschaft $ P$ dieser Struktur aus $ \calK $ heißt
    \begin{itemize}
        \item \underline{Eigenschaft 1. Stufe} (first order property) \underline{in $ \calK $}, falls es einen Satz $ \sigma $ gibt mit:
        \begin{equation*}
            \forall \calA \in \calK : \calA \text{ hat Eigenschaft } P \Leftrightarrow \calA \models \sigma
        \end{equation*}
        \item \underline{allgemeine Eigenschaft 1. Stufe} (general first order property) \underline{in $ \calK $}, falls es eine Menge $ \Sigma $ von Sätzen gibt mit:
        \begin{equation*}
            \forall \calA \in \calK : \calA \text{ hat Eigenschaft } P \Leftrightarrow \calA \models \Sigma
        \end{equation*}
        Wir schreiben $ \calA \models \Sigma :\Leftrightarrow \forall \sigma \in \Sigma: \calA \models \sigma $.
    \end{itemize}
    Ist $ P $ eine Eigenschaft 1. Stufe für die Klasse aller $ L $-Strukturen, so heißt $ P $ auch \underline{endlich axiomatisierbar}.

    \begin{bsp}
        $ L $ beliebig (d.~h. beliebige Signatur).
        Für $ n \in \mathbb{N} $ sei
        \begin{equation*}
            \exists^{\geq n} := (\exists v_1) ... (\exists v_n) (v_1 \not = v_2 \land ... \land v_1 \not = v_n \land v_2 \not = v_3 \land ... \land v_2 \not = v_n \land ... \land v_{n-1} \not = v_n)
        \end{equation*}
        mit $ v_i \not = v_j := \neg (v_i = v_j) $.

        $ \calM \models \exists^{\geq n} \Leftrightarrow | M | \geq n $.
        Also ist die Eigenschaft, mindestens $ n $ Elemente zu haben, eine Eigenschaft 1. Stufe und endlich axiomatisierbar.

        Ebenso $ \exists!^n = \exists^{\geq n} \land \neg \exists^{\geq(n+1)} $, $ \calM \models \exists!^n \Leftrightarrow |M| = n $.

        Sei $ \Sigma = \{ \exists^{\geq n} \mid n \in \mathbb{N} \} $.
        $ \calM \models \Sigma \Leftrightarrow M $ unendlich.
        Die Eigenschaft, unendlich zu sein, ist eine allgemeine Eigenschaft 1. Stufe.
        Aber: Die Eigenschaft, endlich zu sein, ist keine allgemeine Eigenschaft 1. Stufe.
    \end{bsp}

    \begin{bsp}
        Sei $ R $ ein $ 2 $-stelliges Relationsymbol in der Signatur.
        \begin{itemize}
            \item $ \sigma_1 = (\forall v_1) R(v_1, v_2) $ \hfill \textit{(Reflexivität)}
            \item $ \sigma_2 = (\forall v_1) (\forall v_2) \big(R(v_1, v_2) \land R(v_2, v_1)\big) \rightarrow v_1 = v_2 $ \hfill \textit{(Antisymmetrie)}
            \item $ \sigma_3 = (\forall v_1) (\forall v_2) (\forall v_3) \big(R(v_1, v_2) \land R(v_2, v_3)\big) \rightarrow R(v_1, v_3) $ \hfill \textit{(Transitivität)}
        \end{itemize}

        Sei $ \sigma = \sigma_1 \land \sigma_2 \land \sigma_3 $.

        Sei $ \calA = (A, R) \models \sigma \Leftrightarrow R $  ist eine teilweise Ordnung auf $ A $.

        \begin{itemize}
            \item $ \sigma_4 = (\forall v_1) (\forall v_2) R(v_1, v_2) \lor R(v_2, v_1) $ \hfill \textit{(Linearität)}
        \end{itemize}

        $ \calA = (A, R) \models \sigma \land \sigma_4 \Leftrightarrow R $ ist eine lineare Ordnung auf $ A $.

        Die Eigenschaft einer binären Relationsstruktur, eine teilweise geordnete Menge zu sein, bzw. linear geordnet zu sein, ist also endlich axiomatisierbar.

        Sei $ \calA = (A, \leq) $ teilweise geordnet.
        $ \calA $ hat ein größtes Element, falls $ \calA \models \sigma_{gr} $, wobei $ \sigma_{gr} = (\exists y) (\forall x) x \leq y $. \textit{($ x \leq y $ ist Abkürzung für $ \leq(x, y) $)}.
    \end{bsp}

\end{document}
