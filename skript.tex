\documentclass{article}
\usepackage[a4paper]{geometry}
\usepackage{amsmath,amsthm,amssymb}
\usepackage[framemethod=tikz]{mdframed}
\usepackage{etoolbox}
\usepackage[utf8]{inputenc}
\usepackage[ngerman]{babel}
\usepackage{enumitem,nameref}

\title{Skript Logik- und Modelltheorie \\ \large{Nach einer Vorlesung von Prof. Dr. Manfred Droste}}
\author{Felix Linker}
\date{SS2018}

\theoremstyle{definition}
\newtheorem{dfn}{Definition}[section]
\newtheorem{konstr}[dfn]{Konstruktion}
\newtheorem{lem}[dfn]{Lemma}
\newtheorem{satz}[dfn]{Satz}
\newtheorem{kor}[dfn]{Korollar}
\newtheorem*{bsp}{Beispiel}

\theoremstyle{plain}
\newtheorem*{anm}{Anmerkung}
\newtheorem{bem}[dfn]{Bemerkung}

\mdfdefinestyle{standard}{
    skipabove=5pt,
    skipbelow=5pt,
    innerbottommargin=12pt,
    leftmargin=5pt,
    rightmargin=5pt,
    linewidth=1pt,
    roundcorner=4pt,
    backgroundcolor=black!5
}
\mdfdefinestyle{reduced}{
    skipabove=5pt,
    skipbelow=5pt,
    innerbottommargin=12pt,
    leftmargin=5pt,
    rightmargin=5pt,
    linewidth=0.2pt,
    roundcorner=4pt
}
\surroundwithmdframed[style=standard]{dfn}
\surroundwithmdframed[style=standard]{konstr}
\surroundwithmdframed[style=standard]{lem}
\surroundwithmdframed[style=standard]{satz}
\surroundwithmdframed[style=reduced]{kor}
\surroundwithmdframed[style=reduced]{bsp}
\surroundwithmdframed[style=reduced]{anm}
\surroundwithmdframed[style=reduced]{bem}

\newcommand{\m}[1]{\mathcal{#1}}

\newcommand{\sign}[1]{(\m{C}_{#1}, \m{F}_{#1}, \m{R}_{#1}, \sigma'_{#1})}
\newcommand{\struc}[3]{\big(#1, (c^{#2})_{c \in \m{C}_{#3}}, (f^{#2})_{f \in \m{F}_{#3}}, (R^{#2})_{R \in \m{R}_{#3}}\big)}

\sloppy

\begin{document}

    \maketitle

    \nocite{*}
    \bibliography{literatur}
    \bibliographystyle{abbrv}

    \section*{Grundlagen}

    Mathematische Objekte bestehen aus Grundmengen ggf. Relationen, Funktionen und Konstanten.

    Einfache Aussagen betreffen nur Elemente der Grundmenge und haben keine unendlichen Dis- oder Konjunktionen.
    Dies sind prädikatenlogische Aussagen erster Stufe.

    \textit{Monadische Aussagen 2. Stufe erlaubten Quantifizierung über Teilmengen der Grundmenge; Aussagen 2. Stufe erlaubten zusätzlich Quantifizierung über Funktionen und Relationen.}

    \section{Strukturen}

    \begin{dfn}
        Eine \underline{Signatur} $ \sigma $ ist ein Quadrupel
        \begin{equation}
            \sigma = \sign{}
        \end{equation}
        mit einer Menge von Konstantensymbolen $ \m{C} $, einer Menge von Funktionssymbolen $ \m{F} $, einer Menge von Relationensymbolen $ \m{R} $, einer Stelligkeitsfunktion $ \sigma' : \m{F} \cup \m{R} \rightarrow \mathbb{N} $.
    \end{dfn}

    \begin{dfn}
        Eine \underline{Struktur} $ \m{M} $ ist ein Quadrupel
        \begin{equation}
            \m{M} = \struc{M}{\m{M}}{}
        \end{equation}
        mit einer Menge $ M $ (oftmals $ M \neq \emptyset $), Indexmengen $ \m{C}, \m{F}, \m{R} $.
        Wobei gilt $ c^\m{M} \in M $ für $ c \in \m{C} $, $ f^\m{M} : M^{n_f} \rightarrow M $, $ n_f \in \mathbb{N} $ für $ f \in \m{F} $, $ R^\m{M} \subseteq M^{m_R} $ für $ R \in \m{R} $.

        $ \m{M} $ heißt \underline{$ \sigma $-Struktur} bzw. $ \m{M} $ und $ \sigma $ \underline{passen zueinander}, falls:
        \begin{enumerate}
            \item $ n_f = \sigma'(f) $ für $ f \in \m{F} $
            \item $ m_R = \sigma'(R) $ für $ R \in \m{R} $
        \end{enumerate}

        $ c^\m{M} $ heißt auch \underline{Interpretation} von $ c $ in $ \m{M} $; Analoges gilt für $ f^\m{M}, R^\m{M} $.
    \end{dfn}

    \begin{dfn}
        Seien $ \m{M} = \struc{M}{\m{M}}{} $, $ \m{N} = \struc{N}{\m{N}}{} $ zwei $ \sigma $-Strukturen.
        Ein Homomorphismus von $ \m{M} $ nach $ \m{N} $ ist eine Abbildung $ h : M \rightarrow N $ mit:
        \begin{enumerate}
            \item $ h(c^\m{M}) = c^\m{N} $ für allen Konstantensymbole $ c \in \m{C} $
            \item $ h\big(f^\m{M}(a_1, ..., a_n)\big) = f^\m{N}\big(h(a_1), ..., h(a_n)\big) $ für alle Funktionssymbole $ f \in \m{F} $ mit $ n = \sigma'(f) $, $ a_1, ..., a_n \in M $.
            \item \label{itm:homomorphismus-3} $ (a_1, ..., a_n) \in R^\m{M} \Rightarrow \big(h(a_1), ..., h(a_n)) \in R^\m{N} $ für alle Relationensymbole $ R \in \m{R} $, $ n = \sigma'(R) $, $ a_1, ..., a_n \in M $.
        \end{enumerate}

            Für einen Homomorphismus $ h $ von $ \m{M} $ nach $ \m{N} $ schreibt man auch $ h : \m{M} \rightarrow \m{N} $.

            $ h $ heißt \underline{stark}, wenn für alle $ R \in \m{R} $ mit $ n = \sigma'(R) $, $ b_1, ..., b_n \in h(M) \subseteq N $ mit $ (b_1, ..., b_n) \in R^\m{N} $ gilt:
            \begin{equation}
                \label{eq:staerke}
                \exists a_1, ..., a_n \in M : (a_1, ..., a_n) \in R^\m{M}, h(a_i) = b_i, 1 \leq i \leq n
            \end{equation}

            $ h $ heißt \underline{Einbettung}, falls $ h $ stark und injektiv ist.
            Ist $ h $ injektiv, dann lassen sich Bedingung \ref{itm:homomorphismus-3} und Gleichung \eqref{eq:staerke} folgendermaßen zusammenfassen; für alle $ R \in \m{R} $, $ n = \sigma'(R) $, $ a_1, ..., a_n \in M $:
            \begin{equation}
                (a_1, ..., a_n) \in R^\m{M} \Leftrightarrow \big(h(a_1), ..., h(a_n)\big) \in R^\m{N}
            \end{equation}

            $ h $ heißt Isomorphimus, wenn $ h $ eine surjektive Einbettung ist.
            Man schreibt $ \m{M} \cong \m{N} $, wenn ein Isomorphimus $ h : \m{M} \rightarrow \m{N} $ existiert.
            Sind $ h : \m{M} \rightarrow \m{N} $, $ g : \m{N} \rightarrow \m{K} $ Isomorphismen, dann sind auch folgende Funktionen ein Isomorphimus:
            \begin{itemize}
                \item $ h^{-1} $
                \item $ g \circ h : \m{M} \rightarrow \m{K} $
            \end{itemize}

            $ h : \m{M} \Rightarrow \m{M} $ heißt \underline{Automorphismus von $ \m{M} $}, wenn $ h $ ein Isomorphimus ist.
            Die Struktur $ Aut(\m{M}) = \big( \{ h : \m{M} \rightarrow \m{M} \mid h \text{ Automorphismus} \}, \circ \big) $ mit Komposition $ \circ $ ist eine Gruppe.
    \end{dfn}

    \begin{dfn}
        Seien $ \m{M} = \struc{M}{\m{N}}{} $, $ \m{N} = \struc{N}{\m{N}}{} $ $\sigma $-Strukturen.
        $ \m{N} $ heißt \underline{Teil-/Unter-/Substruktur} von $ \m{M} $, falls:
        \begin{enumerate}
            \item $ N \subseteq M $
            \item $ c^\m{N} = c^\m{M} $ für $ c \in \m{C} $
            \item $ f^\m{N}(\bar{a}) = f^\m{M}(\bar{a}) $ für $ f \in \m{F} $, $ \bar{a} \in N^{\sigma'(f)} $
            \item $ \bar{a} \in R^\m{N} \Leftrightarrow \bar{a} \in R^\m{M} $, d.~h. $ R^\m{N} = R^\m{M} \cap N^{\sigma'(R)} $ für $ \bar{a} \in N^{\sigma'(R)} $
        \end{enumerate}

            Wenn $ \m{N} $ eine Teilstruktur von $ \m{M} $ ist, schreibt man auch $ \m{N} \subseteq \m{M} $.
            $ \m{M} $ wird dann auch \underline{Ober- oder Erweiterungsstruktur von $ \m{N} $} genannt.
    \end{dfn}

    \begin{dfn}
        Seien $ \sigma_0 = \sign{0} $, $ \sigma_1 = \sign{1} $ zwei Signaturen.
        Wir definieren $ \sigma_0 \subseteq \sigma_1 $, d.~h. $ \m{C}_0 \subseteq \m{C}_1 $, $ \m{F}_0 \subseteq \m{F}_1 $, $ \m{R}_0 \subseteq \m{R}_1 $ und $ \sigma'_0 = \sigma'_1 \vert_{\m{F}_0 \cup \m{R}_0} $ .

        Sei $ \m{M} = \struc{M}{\m{M}}{1} $ eine $ \sigma_1 $-Struktur.
        Dann heißt $ \m{M} \vert_{\sigma_0} = \struc{M}{\m{M}}{0} $ das \underline{Redukt} von $ \m{M} $ auf $ \sigma_0 $ od. auch $ \sigma_0 $-Redukt.
        Umgekehrt heißt $ \m{M} $ \underline{Expansion} von $ \m{N} := \m{M} \vert_{\sigma_0} $ auf $ \sigma_1 $, d.~h. $ \m{M} $ entsteht aus $ \m{N} $ durch Hinzunahme geeigneter (bzgl. $ \sigma_1 $) Konstanten/Funktionen/Relationen.

        Eine Signatur $ \sigma = \sign{} $ heißt \underline{konstantenlos}, falls $ \m{C} = \emptyset $ und \underline{(rein) relational}, falls $ \m{C} = \m{F} = \emptyset $.
    \end{dfn}

    \begin{bsp}
        Betrachten wir die Gruppe $ \m{G} = (G, 1^\m{G}, \circ^\m{G}) $.
        $ \m{G} $ ist ein Redukt von $ (G, 1^\m{G}, \circ^\m{G}, \circ^{-1\m{G}}) $.
        Von $ \m{R} = (R, 0, 1, +, \cdot) $ ist $ (R, 0, +) $ ein Redukt.
    \end{bsp}

    \section{Sprachen und Formeln}

    Wunsch: Formeln der Form $ \forall x. (x \geq 0 \rightarrow \exists y. x = y \cdot y) $.

    Gegeben eine Signatur $ \sigma = \sign{} $.
    \underline{Grundsymbole} der Sprache $ L = L(\sigma) $ sind:
    \begin{enumerate}
        \item abzählbar viele Variablen $ v_n $, $ n \in \mathbb{N} $
        \item die Symbole von $ \sigma $ auf $ \m{C} \cup \m{F} \cup \m{R} $ (nichtlogische Symbole)
        \item Logisch Zeichen: $ \neg, \land, = $
        \item Quantor $ \exists $
        \item Klammersymbole $ ( $ und $ ) $
    \end{enumerate}

    Die Menge der \underline{Terme} von $ \sigma $ oder $ L(\sigma) $ ist die kleinste Menge für die gilt:
    \begin{enumerate}
        \item Alle Variablen $ v_n $, $ n \in \mathbb{N} $ sind ein Term.
        \item Alle Konstantensymbole $ c \in \m{C} $ sind ein Term.
        \item Wenn $ t_1, ..., t_n $ Terme sind, $ f \in \m{F} $ mit $ \sigma'(f) = n $, dann ist auch $ f(t_1, ..., t_n) $ ein Term.
    \end{enumerate}

    Die Menge der \underline{Atomformeln} bzgl. $ \sigma $ ist die kleinste Menge für die gilt:
    \begin{enumerate}
        \item Wenn $ t_1, t_2 $ Terme sind, dann ist $ t_1 = t_2 $ eine Atomformel,
        \item Wenn $ t_1, ..., t_n $ Terme sind, $ R \in \m{R} $ mit $ \sigma'(R) = n $, dann ist $ R(t_1, ..., t_n) $ eine Atomformel.
    \end{enumerate}

    Die Menge der \underline{Formeln} bzgl. $\sigma $ ist die kleineste Menge für die gilt:
    \begin{enumerate}
        \item Jede Atomformel ist eine Formel.
        \item Wenn $ \varphi, \psi $ Formeln sind, $ x = v_n $ $ n \in \mathbb{N} $ Variable, dann sind auch $ \neg \varphi$, $ (\varphi \land \psi) $, $ (\exists x) \varphi $ Formeln.
    \end{enumerate}

    $ L(\sigma) := $ Menge der Formeln bzgl. $ \sigma $.
    Abkürzungen für zwei Formeln $ \varphi , \psi \in L(\sigma) $ sind wie üblich definiert:
    \begin{itemize}
        \item $ \varphi \lor \psi := \neg (\neg \varphi \land \neg \psi) $
        \item $ \varphi \rightarrow \psi := \neg \varphi \lor \psi $
        \item $ \varphi \leftrightarrow \psi := (\varphi \rightarrow \psi) \land (\psi \rightarrow \varphi) $
        \item $ (\forall x) \varphi := \neg (\exists x) \neg \varphi $
    \end{itemize}

    Die Grundsymbole sind von $ L(\sigma) $ sind absichtlich so spärlich definiert, da Beweise oftmals über den Aufbau von Formeln geführt werden und in diesem Fall minimal-viele Zeichen in Beweisen berücksichtigt werden müssen.

    Seien $ x $ eine Variable $ \varphi $ und eine Formel. Der \underline{(Wirkungs-)Bereich} eines Quantors $ (\exists x) \varphi $ ist definiert als $ \varphi $.
    Ein Vorkommen von $ x $ im Bereich von $ (\exists x) $ heißt \underline{gebunden}.
    Ist $ x $ nicht im Bereich eines Quantors, so heißt dieses Vorkommen \underline{frei}.
    $ x $ ist eine \underline{freie Variable}, wenn $ x $ an mindestens einer Stelle frei vorkommt; ansonsten heißt $ x $ \underline{gebunden}.

    Eine Formel $ \varphi $ heißt \underline{Aussage} oder \underline{Satz}, falls $ \varphi $ keine freien Variablen enthält.

    Nun sei $ \m{M} = \struc{M}{\m{M}}{} $ eine $ \sigma $-Struktur.
    Wir wollen die Gültigkeit von Formeln $ \varphi $ in $ \m{M} $ induktiv definieren.
    Jedes Tupel $ x = (x_i)_{i \in \mathbb{N}} $ mit $ x_i \in M $ ($ x \in M^{\mathbb{N}} $) heißt eine Bewertung der Variablen $ v_i $, $ i \in \mathbb{N} $ von $ L $.
    Ist $ n \in \mathbb{N} $ und $ a \in M $, so sei $ x(n/a) := (x_1, ..., x_{n-1}, a, x_{n+1}, ...) $ die \underline{Aktualisierung} od. \underline{Update} von $ x $ an der Stelle $ n $ durch $ a $.

    \begin{dfn}
        Sei $ \m{M} = \struc{M}{\m{M}}{} $ eine $ \sigma $-Struktur, $ x \in M^{\mathbb{N}} $.
        Der \underline{Wert} eines Terms $ t $ in $ (\m{M}, x) $ sei:
        \begin{enumerate}
            \item $ t^{(\m{M}, x)} := x_n $, falls $ t = v_n $ Variable
            \item $ t^{(\m{M}, x)} := c^\m{M} $, falls $ t = c $ Konstantensymbol
            \item $ t^{(\m{M}, x)} := f^\m{M}(t_1^{(\m{M}, x)}, ..., t_n^{(\m{M}, x)}) $, falls $ t = f(t_1, ..., t_n) $ mit Funktionssymbol $ f \in \m{F} $, $ \sigma'(f) = n $ und Termen $ t_1, ..., t_n $
        \end{enumerate}
    \end{dfn}

    \begin{dfn}
        Wir definieren: $ (\m{M}, x) $ \underline{erfüllt} die Formel $ \varphi $ oder $ \varphi $ \underline{gilt} in $ (\m{M}, x) $ (wir schreiben $ (\m{M}, x) \models \varphi $ od. $ \m{M} \models \varphi[x] $), wie folgt:
        \begin{itemize}
            \item $ (\m{M}, x) \models t_1 = t_2 $, wenn $ t^{(\m{M}, x)}_1 = t^{(\m{M}, x)}_2 $
            \item $ (\m{M}, x) \models R(t_1, ..., t_n) $, wenn $ (t_1^{(\m{M}, x)}, ..., t_n^{(\m{M}, x)}) \in R^\m{M} $
            \item $ (\m{M}, x) \models \varphi \land \psi $, wenn $ (\m{M}, x) \models \varphi $ und $ (\m{M}, x) \models \psi $
            \item $ (\m{M}, x) \models \neg \varphi $, wenn nicht $ (\m{M}, x) \models \varphi $
            \item $ (\m{M}, x) \models (\exists v_n) \varphi $, wenn ein $ a \in M $ mit $ (\m{M}, x(n/a)) \models \varphi $ existiert
        \end{itemize}
    \end{dfn}

    \begin{bem}
        Sei $ \m{M} = \struc{M}{\m{M}}{} $ eine $ \sigma $-Struktur, $ x \in M^{\mathbb{N}} $ eine Bewertung, $\varphi, \psi \in L $.
        \begin{enumerate}
            \item $ (\m{M}, x) \models \varphi \lor \psi \Leftrightarrow (\m{M}, x) \models \varphi $ oder $ (\m{M}, x) \models \psi $\\
            $ (\m{M}, x) \models \varphi \rightarrow \psi \Leftrightarrow $ wenn $ (\m{M}, x) \models \varphi $, dann $ ( \m{M}, x) \models \psi $
            \item $ (\m{M}, x) \models (\forall v_n) \varphi \Leftrightarrow $ für alle $ a \in M gilt: \big(\m{M}, x(n/a)\big) \models \varphi $
            \item Sei $ y \in M^{\mathbb{N}} $ so, dass für alle freien Variablen $ v_n $ in $ \varphi $ gilt: $ x_n = y_n $.
            Dann: $ (\m{M}, x) \models \varphi \Leftrightarrow (\m{M}, y ) \models \varphi $. \label{itm:freie-variablen-gleich}
        \end{enumerate}
    \end{bem}

    \begin{proof}
        ~\par
        \begin{enumerate}
            \item Übungsaufgabe
            \item \begin{align*}
                (\m{M}, x) \models (\forall v_n) \varphi & \Leftrightarrow (\m{M}, x) \models \neg (\exists v_n) \neg \varphi \\
                & \Leftrightarrow \text{nicht: } (\m{M}, x) \models (\exists v_n) \neg \varphi \\
                & \Leftrightarrow \text{nicht ex. } a \in M: \big(\m{M}, x(n/a)\big) \models \neg \varphi \\
                & \Leftrightarrow \text{nicht ex. } a \in M: \text{nicht: } \big(\m{M}, x(n/a)\big) \models \varphi \\
                & \Leftrightarrow \text{für alle } a \in M: \big(\m{M}, x(n/a)\big) \models \varphi
            \end{align*}
            \item Übungsaufgabe; Hinweis: Per Induktion über Formelaufbau.
            \begin{enumerate}
                \item $ \forall t: t^{(\m{M}, x)} = t^{(\m{M}, y)} $ \label{itm:aussage-1}
                \item Induktion: $ \varphi = (t_1 = t_2) $, $ \varphi = R(t_1, ..., t_n) $ mit \ref{itm:aussage-1}, $ \varphi = \psi \land \xi $, $ \varphi = \neg \psi $, $ \varphi = (\exists v_n) \psi $
            \end{enumerate}
        \end{enumerate}
    \end{proof}

    Wegen Bemerkung \ref{itm:freie-variablen-gleich} hängt die Gültigkeit von $ (\m{M}, x) \models \varphi $ \underline{nur} von den ersten $ x_1, ..., x_n $ ab, die alle freien Variablen $ v_i $ in $ \varphi $ überdecken.
    Daher: Wir schreiben $ \m{M} \models \varphi[x_1, ..., x_n] $ falls $ (\m{M}, x) \models \varphi $ und $ n \geq max\{i \mid v_i \text{ frei in } \varphi\} $

    Ist $ \varphi $ ein Satz, dann gilt:
    \begin{align*}
        \m{M} \models \varphi & :\Leftrightarrow \exists x \in M^{\mathbb{N}}: (\m{M}, x) \models \varphi \\
        & \Leftrightarrow \forall x \in M^\mathbb{N}: (\m{M}, x) \models \varphi \\
        & \Leftrightarrow: \varphi \text{ ist wahr in } \m{M}
    \end{align*}

    $ \varphi $ \underline{universell richtig} gdw. $ \forall \sigma$-Strukturen $ \m{M}: \m{M} \models \varphi $.
    \begin{bsp}
        $ (\forall x) x = x $ bzw. $ (\forall v_n) v_n = v_n $
    \end{bsp}

    Zwei $ L $-Strukturen $ \m{M}, \m{N} $ heißen \underline{elementar äquivalent} ($ \m{M} \equiv \m{N} $), falls
    \begin{equation*}
        \forall \text{ Sätze } \sigma: \m{M} \models \sigma \Leftrightarrow \m{N} \models \sigma
    \end{equation*}
    Wobei gilt, dass
    \begin{align}
        & \forall \text{ Sätze } \sigma: \m{M} \models \sigma \Leftrightarrow \m{N} \models \sigma \label{eq:elem-equiv-1} \\
        \Leftrightarrow & \forall \text{ Sätze } \sigma: \m{M} \models \sigma \Rightarrow \m{N} \models \sigma \label{eq:elem-equiv-2}
    \end{align}

    \begin{proof}
        Es gelte \eqref{eq:elem-equiv-2}, wir zeigen \eqref{eq:elem-equiv-1}.
        Sei $ \sigma $ beliebiger Satz.
        $ \m{M} \models \sigma \Rightarrow \m{N} \models \sigma $
        gilt trivialerweise nach Vorbedingung \eqref{eq:elem-equiv-2}.

        Es bleibt zu zeigen, dass $ \m{N} \models \sigma \Rightarrow \m{M} \models \sigma $.
        Es gelte $ \m{N} \models \sigma $.
        Angenommen, es gilt nicht: $ \m{M} \models \sigma $, dann muss gelten, dass $ \m{M} \models \neg \sigma $.
        Offensichtlich ist $ \neg \sigma $ ein Satz.
        \eqref{eq:elem-equiv-2} $ \Rightarrow \m{N} \models \neg \sigma $, also nicht $ \m{N} \models \sigma $. Widerspruch.
    \end{proof}

    Sei $ \m{K} $ eine Klasse von $ L $-Strukturen einer festen Sprache $ L $.
    eine Eigenschaft $ P$ dieser Struktur aus $ \m{K} $ heißt
    \begin{itemize}
        \item \underline{Eigenschaft 1. Stufe} (first order property) \underline{in $ \m{K} $}, falls es einen Satz $ \sigma $ gibt mit:
        \begin{equation*}
            \forall \m{A} \in \m{K} : \m{A} \text{ hat Eigenschaft } P \Leftrightarrow \m{A} \models \sigma
        \end{equation*}
        \item \underline{allgemeine Eigenschaft 1. Stufe} (general first order property) \underline{in $ \m{K} $}, falls es eine Menge $ \Sigma $ von Sätzen gibt mit:
        \begin{equation*}
            \forall \m{A} \in \m{K} : \m{A} \text{ hat Eigenschaft } P \Leftrightarrow \m{A} \models \Sigma
        \end{equation*}
        Wir schreiben $ \m{A} \models \Sigma :\Leftrightarrow \forall \sigma \in \Sigma: \m{A} \models \sigma $.
    \end{itemize}
    Ist $ P $ eine Eigenschaft 1. Stufe für die Klasse aller $ L $-Strukturen, so heißt $ P $ auch \underline{endlich axiomatisierbar}.

    \begin{bsp}
        $ L $ beliebig (d.~h. beliebige Signatur).
        Für $ n \in \mathbb{N} $ sei
        \begin{equation*}
            \exists^{\geq n} := (\exists v_1) ... (\exists v_n) (v_1 \not = v_2 \land ... \land v_1 \not = v_n \land v_2 \not = v_3 \land ... \land v_2 \not = v_n \land ... \land v_{n-1} \not = v_n)
        \end{equation*}
        mit $ v_i \not = v_j := \neg (v_i = v_j) $.

        $ \m{M} \models \exists^{\geq n} \Leftrightarrow | M | \geq n $.
        Also ist die Eigenschaft, mindestens $ n $ Elemente zu haben, eine Eigenschaft 1. Stufe und endlich axiomatisierbar.

        Ebenso $ \exists!^n = \exists^{\geq n} \land \neg \exists^{\geq(n+1)} $, $ \m{M} \models \exists!^n \Leftrightarrow |M| = n $.

        Sei $ \Sigma = \{ \exists^{\geq n} \mid n \in \mathbb{N} \} $.
        $ \m{M} \models \Sigma \Leftrightarrow M $ unendlich.
        Die Eigenschaft, unendlich zu sein, ist eine allgemeine Eigenschaft 1. Stufe.
        Aber: Die Eigenschaft, endlich zu sein, ist keine allgemeine Eigenschaft 1. Stufe.
    \end{bsp}

    \begin{bsp}
        Sei $ R $ ein $ 2 $-stelliges Relationsymbol in der Signatur.
        \begin{itemize}
            \item $ \sigma_1 = (\forall v_1) R(v_1, v_2) $ \hfill \textit{(Reflexivität)}
            \item $ \sigma_2 = (\forall v_1) (\forall v_2) \big(R(v_1, v_2) \land R(v_2, v_1)\big) \rightarrow v_1 = v_2 $ \hfill \textit{(Antisymmetrie)}
            \item $ \sigma_3 = (\forall v_1) (\forall v_2) (\forall v_3) \big(R(v_1, v_2) \land R(v_2, v_3)\big) \rightarrow R(v_1, v_3) $ \hfill \textit{(Transitivität)}
        \end{itemize}

        Sei $ \sigma = \sigma_1 \land \sigma_2 \land \sigma_3 $.

        Sei $ \m{A} = (A, R) \models \sigma \Leftrightarrow R $  ist eine teilweise Ordnung auf $ A $.

        \begin{itemize}
            \item $ \sigma_4 = (\forall v_1) (\forall v_2) R(v_1, v_2) \lor R(v_2, v_1) $ \hfill \textit{(Linearität)}
        \end{itemize}

        $ \m{A} = (A, R) \models \sigma \land \sigma_4 \Leftrightarrow R $ ist eine lineare Ordnung auf $ A $.

        Die Eigenschaft einer binären Relationsstruktur, eine teilweise geordnete Menge zu sein, bzw. linear geordnet zu sein, ist also endlich axiomatisierbar.

        Sei $ \m{A} = (A, \leq) $ teilweise geordnet.
        $ \m{A} $ hat ein größtes Element, falls $ \m{A} \models \sigma_{gr} $, wobei $ \sigma_{gr} = (\exists y) (\forall x) x \leq y $. \textit{($ x \leq y $ ist Abkürzung für $ \leq(x, y) $)}.
    \end{bsp}

    \begin{bsp}[Gruppen]
        Z.B. $ \sigma = \big(1, (\cdot, inv), \emptyset, \sigma'\big) $ mit $ \sigma'(\cdot) = 2 $, $ \sigma'(inv) = 1 $.
        Sei $ \sigma_{\m{G}} $ die Konjunktion von:
        \begin{enumerate}
            \item $ \forall x, y, z: (x \cdot y) \cdot z = x \cdot ( y \cdot z) $ \hfill \textit{(Assoziativität)}
            \item $ \forall x: x \cdot 1 = x $ \hfill \textit{(Rechts Eins)}
            \item $ \forall x: x \cdot inv(x) = 1 $ \hfill \textit{(Inverse)}
        \end{enumerate}
        $ \m{G} = \big(G, 1, (\cdot, inv), \emptyset\big) $ ist Gruppe gdw. $ \m{G} \models \sigma_{\m{G}} $.
        $ \m{G} $ ist abelsche Gruppe gdw. $ \m{G} \models \sigma_{C\m{G}} $, wobei $ \sigma_{C\m{G}} = \sigma_{\m{G}} \land \forall x \forall y: x \cdot y = y \cdot x $.
        Falls $ \sigma_1 = (1, \cdot, \emptyset, \sigma_1') $ mit $ \sigma_1'(\cdot) = 2 $, so ersetze den 3. Satz durch $ \forall x \exists x: x \cdot y = 1 $ \textit{(Existenz von Rechtsinversen)}.
        Neue Konjunktion: $ \sigma_{\m{G}}^1 $; $ \m{G} $ Gruppe gdw. $ \m{G} \models \sigma_{\m{G}}^1 $.
    \end{bsp}

    \begin{bsp}[Ringe]
        $ \sigma = \big((0, 1), (+, \underbrace{-}_{\text{inv. bzgl. }+}, \cdot), \emptyset, \sigma' \big) $, mit $ \sigma'(+) = \sigma'(\cdot) = 2 $ und $ \sigma'(-) = 1 $.
        Sei $ \sigma_{C\m{G}}' $ wie $ \sigma_{C\m{G}} $ mit folgenden Ersetzungen:
        \begin{itemize}
            \item $ \cdot $ wird ersetzt durch $ + $
            \item $ 1 $ wird ersetzt durch $ 0 $
            \item $ inv $ wird ersetzt durch $ - $
        \end{itemize}
        Sei $ \sigma_{\m{R}} $ die Konjunktion von $ \sigma_{C\m{G}}' $ mit:
        \begin{enumerate}
            \item $ \forall x, y, x : (x \cdot y) \cdot z = x \cdot (y \cdot z) $
            \item $ \forall x: x \cdot 1 = x \land 1 \cdot x = x $
            \item $ \forall x, y, z: (x + y) \cdot z = (x \cdot z) + (y \cdot z) $ \hfill \textit{(Rechtsdistributivität)}
            \item $ \forall x, y, z: z \cdot (x + y) = (z \cdot x) + (z \cdot y) $ \hfill \textit{(Linksdistributivität)}
        \end{enumerate}
        $ \m{R} $ ist ein Ring gdw. $ \m{R} \models \sigma_{\m{R}} $.
        $ \m{R} $ ist ein kommutativer Ring gdw $ \m{R} \models \sigma_{C\m{R}} $ mit $ \sigma_{C\m{R}} = \sigma_{\m{R}} \land (\text{``$\;\cdot$'' ist kommutativ}) $.
    \end{bsp}

    \begin{bsp}[Körper]
        % F steht für Feld; in der alten englischen Literatur werden Körper noch als Feld bezeichnet
        Sei $ \sigma_F $ die Konjunktion von $ \sigma_{C\m{R}} $ mit
        \begin{enumerate}
            \item $ \neg(0 = 1) $
            \item $ \forall x: \big( \neg( x = 0 ) \rightarrow \exists y: x \cdot y = 1 \big) $
        \end{enumerate}
        $ \m{K} $ ist ein Körper gdw. $ \m{K} \models \sigma_F $.

        Sei $ p \in \mathbb{N} $.
        Ein Körper $ \m{K} $ hat Charakteristik $ p $, falls $ p $ die kleinste natürliche Zahl ist mit $ p \cdot 1 = 0 $, wobei $ p \cdot 1 := \Big( ... \underbrace{\big((1 + 1) + 1 \big) + ... + 1}_{p \text{ Einsen}} \Big) $.
        Falls $ p $ existiert, so ist $ p $ eine Primzahl.
        Sonst: $ char(\m{K}) = 0 $.
        \underline{Beispiel}: $ (\mathbb{Z}/p\mathbb{Z}, +, \cdot, 0, 1) $.

        \begin{anm}
            $ \mathbb{Z}/p\mathbb{Z}, +, \cdot ) $ ist Körper gdw. $ p $ ist eine Primzahl.
        \end{anm}

        Sei $ p $ eine Primzahl, $ n \in \mathbb{N} $.
        \begin{enumerate}%[label*=\alph*)]
            \item Bis auf Isomorphie existiert genau ein Körper mit $ p^n $ Elementen.
            \item Wenn $ \m{K} $ ein endlicher Körper ist, dann ex. Primzahl $ p $ und $ n \in \mathbb{N} $ mit $ | \m{K} | = p^n $ und $ char(\m{K}) = p $.
            Es gibt auch unendliche Körper der Charakteristik $ p $.
        \end{enumerate}
        Die Eigenschaft, ein Körper der Charakteristik $ p $ zu sein, ist eine Eigenschaft 1. Stufe.
        Die Eigenschaft, ein Körper der Charakteristik  0 zu sein ist eine allgemeine Eigenschaft 1. Stufe, beschreibbar durch $ \{ \sigma_F \} \cup \big\{ \neg (p \cdot 1 = 0) \mid p \in \mathbb{N} \big\} $.
        Aber letzteres ist \underline{keine} Eigenschaft 1. Stufe.
    \end{bsp}

    \begin{bsp}[Vektorräume]
        Gegeben Körper $ \m{K} $.
        \begin{description}
            \item[1. Möglichkeit]
            $ \m{V} = \big(V, 0, +, (r_k)_{k \in \m{K}} \big) $ mit:
            \begin{itemize}
                \item $ (V, 0, +) $ ist abelsche Gruppe
                \item $ \forall k \in \m{K}: r_k : V \rightarrow V $ Funktion (``Multiplikation'' mit k von rechts)
                \item Übliche Gesetze fordern (Assoziativität, Distributivität, ...); wir schreiben $ x \cdot k := r_k(x) $
                \item $ \forall x: (x \cdot k) \cdot k' = x \cdot (k \cdot k') $ Abkürzung für $ r_{k'}\big(r_k(x)\big) = r_{k \cdot k'}(x) $.
                \item $ \forall x, y: (x + y ) \cdot k = (x \cdot k) + (y \cdot k) $
                \item $ 0 \cdot k = 0 $, $ \forall x: x \cdot 0 = 0 \land x \cdot 1 = x $
            \end{itemize}

            \item[2. Möglichkeit]
            $ \m{V} = \big(X, (0_V, 0_K, 1_K), +, \cdot, V, K\big) $ mit:
            \begin{itemize}
                \item $ V , K $ ein-stellige Relationen
                \item $ X = V \; \dot\cup \; K $
                \item $ 0_V \in V, 0_k, 1_K \in K $
                \item $ +, \cdot : X \times X \rightarrow X $ Abbildung
                \item Für alle Element, die zu $ V $ gehören, die Axiome der abelschen Gruppe fordern.
                \item Für alle Element, die zu $ K $ gehören, die Körperaxiome fordern.
                \item $ \forall x, y, z: (V(x) \land K(y) \land x \cdot y = z) \rightarrow V(z) $
                \item Für Elemente aus $ V $ bzw. $ K $ die üblichen Gesetze (d.~h. Assoziativität, Distributivität, usw.) fordern.
            \end{itemize}
            Somit ist die Eigenschaft, ein $ K $-Vektorraum zu sein, endlich axiomatisierbar.

            \item[3. Möglichkeit]
            Über eine ``zwei-sortige'' Struktur $ \m{V} = \big(V, K, (0_V, 0_K, 1_K), (+_V, +_K, \cdot_K, \cdot)\big) $ mit Konstanten, Funktionen $ +_V, +_K, \cdot_K $ für ede Sorte (=Grundmenge) $ V, K $ und $ \cdot : V \times K \rightarrow $ mit der Forderung üblicher Gesetze.

            Diese Definition ist jedoch nicht mit der Definition von Strukturen, wie wir sie eingeführt haben, vereinbar.
        \end{description}
    \end{bsp}

    \begin{bsp}[Boole'sche Algebren]
        ~\par
        \begin{description}
            \item[1. Möglichkeit]
            \label{itm:boolesche-algebra-struktur}
            $ \m{B}= (B, 0, 1, \lor, \land, *) $ mit $ \sigma'(\lor) = \sigma'(\land) = 2 $ und $ \sigma'(*) = 1 $.
            $ \m{B} $ ist Boole'sche Algebra, wenn gilt:
            \begin{itemize}
                \item $ \forall x, y, z: (x \land y) \land z = x \land (y \land z) $ bzw. $ (x \lor y) \lor z = x \lor (y \lor z) $ \hfill \textit{(Assoz.)}
                \item $ \forall x, y: x \land y = y \land x $ bzw. $ x \lor y = y \lor x $ \hfill \textit{(Kom.)}
                \item $ \forall x, y, z: (x \land y) \lor z = (x \lor z) \land (y \lor z) $ bzw. $ (x \lor y) \land z = (x \land z) \lor (y \land z) $ \hfill \textit{(Distr.)}
                \item $ \forall x, y: (x \land y) \lor y = y $ bzw. $ (x \lor y) \land y = y $ \hfill \textit{(Absorption)}
                \item $ \forall x: x \land x^* = 0 $ bzw. $ x: x \lor x^* = 1 $ \hfill \textit{(Komplemente)}
                \item $ x^{**} = x $
                \item $ (x \land y)^* = x^* \lor y^* $ bzw. $ (x \lor y)^* = x^* \land y^* $ \hfill \textit{(de'Morgan)}
            \end{itemize}
            \begin{bsp}
                $ \m{B} = \big(\{0, 1\}, 0, 1, \lor, \land, \neg\big) $.
                Oder für eine Menge $ M $ die Struktur $ \big(\m{P}(M), \emptyset, M, \cup, \cap, \bar\; \big) $.
            \end{bsp}

            \item[2. Möglichkeit]
            \label{itm:boolesche-algebra-ordnung}
            $ \m{B} = (B, \leq) $ teilweise Ordnung.
            Für $ A \subseteq B $, $ x \in B $ gilt
            \begin{enumerate}[series=teilweise-ordnung-voraussetzung]
                \item $ x \leq A \Leftrightarrow x $ ist untere Schranke von $ A  \Leftrightarrow \forall a \in A: x \leq a $
                \item $ x \geq A \Leftrightarrow x $ ist obere Schranke von $ A \Leftrightarrow \forall a \in A: x \geq a $
                \item $ x = inf(A) \Leftrightarrow x $ ist die größte untere Schranke von $ A \Leftrightarrow (x \leq A) \land \forall y: (y \leq A \rightarrow y \leq ) $
                \item $ x = sup(A) \Leftrightarrow x $ ist die kleinste obere Schranke von $ A \Leftrightarrow (A \leq x) \land \forall y: (A \leq y \rightarrow x \leq y) $
            \end{enumerate}

            Wir schreiben $ x = y \lor z $ als Abkürzung für
            \begin{equation*}
                y \leq x \land z \leq x \land \forall w: (y \leq w \land z \leq w) \rightarrow x \leq w
            \end{equation*}
            Und wir schreiben $ x = y \land z $ für
            \begin{equation*}
                x \leq y \land x \leq z \land \forall w: (w \leq y \land w \leq z) \rightarrow w \leq x
            \end{equation*}

            \begin{enumerate}[resume*=teilweise-ordnung-voraussetzung]
                \item $ A = \{y, z\}: x = y \lor z \text{ für } x = sup\big(\{y, z\}\big) $ bzw. $ x = y \land z \text{ für } x = inf\big(\{y, z\}\big) $
            \end{enumerate}

            Desweiteren fordern wir:
            \begin{description}
                \item[Distributivität]
                \begin{equation}
                \label{eq:boolesch-distributiv}
                    \begin{aligned}
                        \forall x, y, z, u, v, w, p: & \big((u = x \land y) \land (v = x \lor z) \land (w = y \lor z) \land (p = v \land w)\big) \\
                        & \rightarrow p = u \lor z
                    \end{aligned}
                \end{equation}
                (Dies drückt unter Verwendung der obigen Abkürzung aus, dass $ (x \land y) \lor z = (x \lor z) \land (y \lor z) $.)
            \end{description}

            $ (B, \leq) $ ist ein Verband gdw. $ \forall x, y \exists w, z: w = x \land y, z = x \lor z $.

            $ (B, \leq) $ ist ein distributiver Verband gdw. $ (B, \leq) $ ist ein Verband und es es gilt \eqref{eq:boolesch-distributiv}.

            \begin{anm}
                Sei $ (B, \leq) $ ein Verband.
                Betrachten wir die Sätze
                \begin{equation}
                \label{eq:verband-equiv-1}
                    \forall x, y, z \in B: (x \land y) \lor z = (x \lor z) \land (y \lor z)
                \end{equation}
                \begin{equation}
                \label{eq:verband-equiv-2}
                    \forall x, y, z \in B: (x \lor y) \land z = (x \land z) \lor (y \land z)
                \end{equation}
                Es gilt dann, dass \eqref{eq:verband-equiv-1} gdw. \eqref{eq:verband-equiv-2}.
            \end{anm}

            \begin{description}
                \item[Kleinste/Größte Element]
                \begin{equation*}
                    \exists 0,1 \in B : \forall x: 0 \leq x \land y \leq 1
                \end{equation*}

                \item[Komplemente in Verbänden]
                \begin{equation*}
                    \forall x \exists y: (x \land y) = 0 \land (x \lor y = 1)
                \end{equation*}
            \end{description}

            $ (B, \leq) $ erfüllt diese Sätze mit $ 0 \not = 1 $ gdw $ (B, \leq) $ ist Boole'sche Algebra.
        \end{description}

        \paragraph{Äquivalenz der Möglichkeiten}
        Wir betrachten nun, dass beide vorgestellten Möglichkeiten der Darstellung einer Boole'schen Algebra äquivalent sind.

        \underline{Hinrichtung}: Sei $ \m{B} = (B, 0, 1, \lor, \land, * ) $ eine Boole'sche Algebra (im ersten Sinn).
        Definiere $ \leq $ auf $ B $ durch:
        \begin{align*}
            x \leq y & :\Leftrightarrow x \lor y = y \\
            \text{bzw. äquivalent } & :\Leftrightarrow x \land y = x
        \end{align*}
        Dann ist $ (B, \leq) $ eine Boole'sche Algebra im 2. Sinn und
        \begin{align*}
            \forall x, y, z \in B: & x \lor y = z \Leftrightarrow z = sup \big(\{x, y\}\big) \\
            \text{bzw. } & x \land y = z \Leftrightarrow z = inf \big(\{x, y\}\big)
        \end{align*}

        \underline{Rückrichtung}: Sei $ (B, \leq) $ eine Boole'sche Algebra im 2. Sinn mit kleinstem Element $ 0 $, größtem Element $ 1 $ und $ 0 \not = 1 $.
        Wir definieren $ \lor : B \times B \rightarrow B $ als
        \begin{equation*}
            x \lor y := sup\big(\{x, y\}\big)
        \end{equation*}
        sowie $ \land : B \times B \rightarrow B $ als
        \begin{equation*}
            x \land y := inf\big(\{x, y\}\big)
        \end{equation*}
        Wir definieren $ * : B \rightarrow B $ als $ x^* := y $, wenn $ y \in B $ mit
        \begin{equation*}
            x \land y = 0 \text{ sowie } x \lor y = 1
        \end{equation*}

        Dann ist $ (B, 0, 1, \lor, \land, *) $ eine Boole'sche Algebra im 1. Sinn und
        \begin{equation*}
            x \lor y = y \Leftrightarrow y \leq y
        \end{equation*}

        $ * $ ist eine wohldefinierte Abbildung aufgrund der folgenden Anmerkung:
        \begin{anm}
            Sei $ (B, \leq) $ distributiver Verband mit $ 0, 1 $.
            Dann gilt, falls $ x, y, y' \in B $ und $ x \land y = 0 $, $ x \lor y = 1 $, $ x \land y' = 0 $, $ x \lor y' = 1 $, dann ist $ y = y' $.
            D.~h. Komplemente sind (falls sie existieren) eindeutig.
        \end{anm}

        \begin{bsp}
            Für Boole'sche Algebra im 2. Sinn:
            \begin{enumerate}
                \item $ \big(\{0,1\}, \leq \big) $
                \item $ \big(\m{P}(M), \subseteq) $, $ M \not = \emptyset $ mit:
                \begin{align*}
                    X ,Y \in \m{P}(M): & \; X \lor Y = X \cup Y \\
                    \text{und } & \; X \land Y = X \cap Y
                \end{align*}
            \end{enumerate}
        \end{bsp}
    \end{bsp}

    Sei $ \Sigma $ eine Menge von Sätzen für Signatur $ \sigma $.
    $ \Sigma $ hat ein Modell, falls gilt, es ex. eine $ \sigma $-Struktur $ \m{M} $ mit $ \m{M} \models \Sigma $, d.~h. $ \forall \sigma' \in \Sigma: M \models \sigma' $.
    Eine Struktur $ \m{M} = (M,...) $ heißt \underline{endlich}, falls gilt, dass $ M $ endlich ist.

    \begin{satz}[Kompaktheitssatz]\label{satz:kompaktheitssatz}
        Sei $ \Sigma $ eine beliebige Menge von Sätzen zur Signatur $ \sigma $.
        Dann gilt: $ \Sigma $ hat ein Modell gdw. jede endliche Teilmenge von $ \Sigma $ hat ein Modell:
        \begin{equation*}
            M \models \Sigma \Leftrightarrow \forall \sigma \in \Sigma: M \models \Sigma
        \end{equation*}
    \end{satz}

    \begin{kor}\label{kor:unendliches-modell}
        Sei $ \Sigma $ eine Menge von Sätzen.
        Falls $ \Sigma $ beliebig große endliche Modelle hat, so hat $ \Sigma $ auch ein unendliches Modell.
    \end{kor}

    \begin{proof}
        Sei $ \Sigma^+ = \Sigma \cup \{ \exists^{\geq n} \mid n \in \mathbb{N} \} $.
        Nach Vorbedingung für $ \Sigma $ gilt, dass $ \forall E \subseteq \mathbb{N} : E \text{ endlich } \rightarrow E \cup \{ \exists^{\geq n} \mid n \in E \} $ hat ein Modell.
        Somit können wir Satz \ref{satz:kompaktheitssatz} auf $ \Sigma^+ $ anwenden, also hat $ \Sigma^+ $ ein Modell $ M $.
        Somit gilt $ M \models \Sigma $ und $ M $ ist unendlich.
    \end{proof}

    \begin{kor}
        Die Eigenschaft, eine endliche Menge zu sein, ist keine endlich allgemeine Eigenschaft 1. Stufe.
    \end{kor}

    \begin{proof}
        Angenommen, $ \Sigma $ wäre eine Menge von Sätzen mit $ a \models \Sigma \Leftrightarrow a $ ist endlich.
        $ \Sigma $ hat beliebig große, endliche Modelle (Mengen mit beliebigen Konstanten, Funktionen, Relationen).
        Nach Korollar \ref{kor:unendliches-modell} hat $ \Sigma  $ ein unendliches Modell $ a $. Widerspruch.
    \end{proof}

    \paragraph{Körper}
    $ \Delta := \{ \sigma_F \} \cup \{ \neg(\overbrace{p \cdot 1}^{1 + ... + 1} = 0 ) \mid p \in \mathbb{N} \} $.
    $ a \models \Delta $ gdw. $ a $ ist Körper der Charakteristik $ 0 $.
    Also ist die Eigenschaft, ein Körper der Charakteristik 0 zu sein, eine allgemeine Eigenschaft 1. Stufe.

    \begin{satz}
        Die Eigenschaft, ein Körper der Charakteristik $ 0 $ zu sein, ist keine Eigenschaft 1. Stufe.
    \end{satz}

    \begin{proof}
        Angenommen, es gäbe einen Satz $ \sigma $ mit: $ a \models \sigma \Leftrightarrow a $ ist Körper der Charakteristik $ 0 $.

        Betrachte $ \Delta' := \underbrace{\Delta}_{\text{s.o.}} \cup \{ \neg \sigma \} $.
        Jede endliche Teilmenge von $ \Delta' $ hat ein Modell (einen Körper der Charakteristik $ p $, wobei $ p $ geeignete Primzahl ist).
        Nach Satz \ref{satz:kompaktheitssatz} hat $ \Delta' $ ein Modell $ a $, also $ a \models \Delta $ und $ a \models \neg \sigma $.
        Widerspruch.
    \end{proof}

    \section{Elementare Substrukturen}

    \begin{dfn}
        Seien $ \m{M} $, $ \m{N} $ zwei $ \sigma $-Strukturen.
        $ \m{M} $ heißt \underline{elementare Substruktur (elementares Submodell)} von $ \m{N} $ gdw. $ M \subseteq N $ und für alle Formeln $ \varphi $ und Bewertungen $ x \in M^\mathbb{N} $ gilt:
        \begin{equation*}
            (\m{M}, x) \models \varphi \Leftrightarrow (\m{N}, x) \models \varphi
        \end{equation*}
        Das heißt:
        \begin{equation*}
            \forall \varphi = \varphi(v_1, ..., v_n) \forall x_1, ..., x_n \in M: \m{M} \models \varphi[x_1, ..., x_n] \Leftrightarrow \m{N} \models \varphi[x_1, ..., x_n]
        \end{equation*}

        Wir schreiben $ \m{M} \preceq \m{N} $, wenn $ \m{M} $ elementare Substruktur von $ \m{N} $ ist.
        Eine Abbildung $ h : M \rightarrow N $ heißt \underline{elementare Einbettung}, falls:
        \begin{equation*}
            \forall \varphi = \varphi(v_1, ..., v_n) \forall x_1, ..., x_n \in M:
            M \models \varphi[x_1, ..., x_n] \Leftrightarrow N \models \varphi[h(x_1), ..., h(x_n)]
        \end{equation*}

        Da $ \m{M} \subseteq \m{N} \Leftrightarrow M \subseteq N \land \forall \text{Atomformeln } \varphi \forall x \in M^\mathbb{N}: (\m{M}, x) \models \varphi \Leftrightarrow (\m{N}, x) \models \varphi $ (Beweis in Übung), gilt $ \m{M} \preceq \m{N} \Rightarrow \m{M} \subseteq \m{N} $.

        Es ist klar, dass $ \m{M} \preceq \m{N} \Leftrightarrow M \subseteq N $ und $ id: M \rightarrow N $ ist eine elementare Einbettung.

        Üben:
        \begin{enumerate}
            \item Wenn $ h: M \rightarrow N $ elementare Einbettung ist, dann ist $ h : \m{M} \rightarrow \m{N} $ eine Einbettung.

            \item Wenn $ h : \m{M} \rightarrow \m{N} $ elementare Einbettung ist, dann ist $ h(\m{M}) := \m{N} |_{h(M)} \preceq \m{N} $.
            Wenn darüber hinanus $ h $ injektiv ist, gilt dann die Umkehrung?

            \item wenn $ h : \m{M} \rightarrow \m{N} $ ein Isomorphimus ist, dann ist $ h $ eine elementare Einbettung.
        \end{enumerate}

        Wenn eine elementare Einbettung $ h: \m{M} \rightarrow \m{N} $ ex., dann gilt $ \underbrace{\m{M} \equiv \m{N}}_{\text{nur Sätze betrachtet}} $.

        Wenn $ \m{M} \subseteq \m{N} $ und $ \m{M} \equiv \m{N} $, dann \underline{nicht} $ \m{M} \preceq \m{N} $.
    \end{dfn}

    \begin{bsp}
        Seien $ \m{M} = (\mathbb{N}_+, \leq) $, $ \m{N} = (\mathbb{N}_0, \leq) $.
        Es gilt trivialerweise $ \m{M} \subseteq \m{N} $.
        Es gilt $ \m{M} \cong \m{N} $, da $ n \mapsto n - 1 $ ein Isomorphimus ist, also $ \m{M} \equiv \m{N}$.
        Aber $ \m{M} \not \preceq \m{N} $, denn für $ \varphi(x) = \exists y: y < x $ gilt $ \m{N} \models \varphi(1) $, aber $ \m{M} \not \models \varphi(1) $.
    \end{bsp}

    Elementare Substrukturen sind wichtig, um $ \m{M} \equiv \m{N} $ über den Umweg $ \m{M} \preceq \m{N} $ zu zeigen: Beweise funktionieren häufig durch Induktion über Formelaufbau (nicht Sätze), also werden Aussagen über Formeln benötigt.

    \begin{lem}
    \label{lem:elem-substr}
        Sei $ \m{M} \subseteq \m{N} $.
        \begin{align*}
            \m{M} \preceq \m{N} \Leftrightarrow & \forall \text{Formeln }\varphi \forall x \in M^\mathbb{N} \forall v_n: \\
            & \bigg( \underbrace{(\m{N}, x) \models (\exists v_n)\varphi}_{\exists b \in N: \big(\m{N}, x(n/b)\big) \models \varphi} \Leftrightarrow \exists a \in M : \big(\m{N}, x(n/a)\big) \models \varphi \bigg)
        \end{align*}
    \end{lem}

    \begin{proof}
        ~\par
        \begin{description}
            \item[``$ \Rightarrow $''] Sei $ \m{M} \preceq \m{N} $.
            \begin{description}
                \item[``$ \Rightarrow $'']
                Sei $ \varphi $ Formel, $ x \in M^\mathbb{N} $, $ v_n $ eine Variable.
                \begin{align*}
                    (\m{N}, x) \models (\exists v_n) \varphi \Leftrightarrow & (\m{M}, x) \models (\exists v_n) \varphi & \m{M} \preceq \m{N} \\
                    \Leftrightarrow & \exists a \in M: \bigg( \big(\m{M}, x(n/a)\big) \models \varphi \stackrel{\m{M} \preceq \m{N}}{\Rightarrow} \big(\m{N}, x(n/a)\big) \models \varphi \bigg)
                \end{align*}
                \item[``$ \Leftarrow $''] trivial, da $ a \in M \subseteq N $.
            \end{description}
            \item[``$ \Leftarrow $''] z.Z.:
            \begin{equation}
                \label{eq:elem-substr-zz}
                \forall \varphi \forall x \in M^\mathbb{N}: (\m{M}, x) \models \varphi \Leftrightarrow (\m{N}, x) \models \varphi \equiv \m{M} \preceq \m{N}
            \end{equation}
            Beweis durch Induktion über Formelauf.

            \begin{description}
                \item[IA] Sei $ \varphi $ eine Atomformel. \eqref{eq:elem-substr-zz} klar wegen $ \m{M} \subseteq \m{N} $ (+ Übungsaufgabe).

                \item[IS] Angenommen $ \psi, \chi $ sind Formeln, für die \eqref{eq:elem-substr-zz} gilt.
                Dann ist klar, dass \eqref{eq:elem-substr-zz} für $ \neg \psi $ und $ \psi \land \chi $ gilt.

                z.Z. \eqref{eq:elem-substr-zz} gilt für $ (\exists v_n) \psi $.
                \begin{align*}
                    (\m{M}, x) \models (\exists v_n) \psi \Leftrightarrow & \exists a \in M: \underbrace{\big(\m{M}, x(n/a)\big) \models \psi}_{\Leftrightarrow (\m{N}, x(n/a) \models \psi) \text{ (IV)}} \\
                    \Leftrightarrow & \exists a \in N: \big(\m{N}, x(n/a)\big) \models \psi & M \subseteq N \text{ bzw. Voraussetzung} \\
                    \Leftrightarrow & (\m{N}, x) \models (\exists v_n) \psi
                \end{align*}
            \end{description}
        \end{description}
    \end{proof}

    \begin{kor}
    \label{kor:elem-substr}
        Sei $ \m{M} \subseteq \m{N} $.
        \begin{align*}
            \m{M} \preceq \m{N} \Leftrightarrow & \forall \varphi = \varphi(v_0, ..., v_n) \forall a_0, ..., a_{n - 1} \in M: \\
            & \exists b \in N: \m{N} \models \varphi[a_0, ..., a_{n - 1}, b] \Rightarrow \exists a \in M: \m{N} \models \varphi[a_0, ..., a_{n - 1}, a]
        \end{align*}
    \end{kor}

    \begin{proof}
        Trivial nach Lemma \ref{lem:elem-substr}.
    \end{proof}

    \begin{bsp}
        Wir betrachten $ (\mathbb{Q}, \leq) \preceq (\mathbb{R}, \leq) $.
        Aber $ (\mathbb{Q}, \leq) \not \cong (\mathbb{R}, \leq) $, da $ | \mathbb{Q} | \neq | \mathbb{R} | $ oder, da $ (\mathbb{R}, \leq) $ Dedekind-vollständig, aber $ (\mathbb{Q}, \leq) $ nicht Dedekind-vollständig ist.

        \begin{anm}
            \begin{equation*}
                Q = (\mathbb{Q}, +, \cdot, 0, 1) \not \equiv (\mathbb{R}, +, \cdot, 0, 1) = R
            \end{equation*}
            Sei $ \sigma = (\exists x) x \cdot x = 1 + 1 $, dann $ R \models \sigma $, aber $ Q \not \models \sigma $.
        \end{anm}

        \begin{proof}
            $ (\mathbb{Q}, \leq) \subseteq (\mathbb{R}, \leq) $ trivial.
            Sei $ \varphi = \varphi(v_0, ..., v_n) $ Formel, $ a_0, ..., a_{n - 1} \in \mathbb{Q} $, $ b \in \mathbb{R} $ mit $ (\mathbb{R}, \leq) \models \varphi[a_0, ..., a_{n - 1}, b] $.
            Nach Korollar \ref{kor:elem-substr}.

            z.Z. $ \exists a \in \mathbb{Q}: (\mathbb{R}, \leq) \models \varphi[a_0, ..., a_{n - 1}, a] $.
            O.B.d.A. gelte $ a_0 < a_1 < ... < a_{n - 1} $ und $ b \in \mathbb{R} \setminus \mathbb{Q} $.
            Sei etwa $ a_k < b < a_{k+1} $ mit $ k \in \{0, ..., n - 2\} $.
            Wähle $ a \in \mathbb{Q} $ mit $ a_k < a < a_{k + 1 } $.

            Definiere $ h: \mathbb{R} \rightarrow \mathbb{R} $
            \begin{equation*}
                x \mapsto \begin{cases}
                    x & x \leq a_k \lor a_{k + 1} \leq x \\
                    \frac{a - a_k}{b - a_k} \cdot (c - a_k) + a_k & a_k \leq x \leq b \\
                    \frac{a_{k + 1} - a}{a_{k + 1} - b} \cdot (x - b) + a & b \leq x \leq a_{k + 1}
                \end{cases}
            \end{equation*}

            Daraus folgt, dass $ h: (\mathbb{R}, \leq) \rightarrow (\mathbb{R}, \leq) $ ist ein Isomorphimus mit $ h(a_i) = a_i $ für $ i = 0, ..., n - 1 $.
            Somit ist $ h $ eine elementare Einbettung.
            Zusammen mit $ (\mathbb{R}, \leq) \models \varphi[a_0, ..., a_{n - 1}, b] $ folgt dann, dass
            \begin{align*}
                (\mathbb{R}, \leq) \models & \varphi[h(a_0), ..., h(a_{n - 1}), h(b)] \\
                = & \varphi[a_0, ..., a_{n - 1}, a]
            \end{align*}
        \end{proof}
    \end{bsp}

    \begin{anm}
        Beachte, wenn $ \m{M}_1 \subseteq \m{M}_2 $ und $ \m{M}_2 \subseteq \m{M}_3 $, dann $ \m{M}_1 \subseteq \m{M}_3 $. (s. Übung).
        Und wenn $ \m{M}_1 \preceq \m{M}_2 $ und $ \m{M}_2 \preceq \m{M}_3 $, dann $ \m{M}_1 \preceq \m{M}_3 $.
    \end{anm}

    Sei $ (\alpha, \leq) $ eine geordnete Menge.
    Eine Folge $ (\m{M}_i)_{i \in \alpha} $ von $ \bar{\sigma} $-Strukturen heißt \underline{Kette}, falls
    \begin{equation*}
        \forall i, j \in \alpha: i < j \Rightarrow \m{M}_i \subseteq \m{M}_j
    \end{equation*}

    Sei nun $ (\m{M}_i)_{i \in \alpha} $ eine Kette und $ \m{M}_i = \struc{M_i}{\m{M}_i}{} $.
    Insbesondere $ c^{\m{M}_i} = c^{\m{M}_j} $ für alle $ c \in \m{C} $.
    Setze
    \begin{itemize}
        \item $ M := \bigcup_{i \in \alpha} M_i $
        \item $ c^\m{M} := c^{\m{M}_i} $ \hfill ($ c \in \m{C}, i \in \alpha $ beliebig)
        \item $ R^\m{M} := \bigcup_{i \in \alpha} R^{\m{M}_i} $ \hfill ($ R \in \m{R} $)
        \item $ f^\m{M} := \bigcup_{i \in \alpha} f^{\m{M}i} $ \hfill ($ f \in \m{F} $)
    \end{itemize}

    Dann gilt $ \m{M} := \struc{M}{\m{M}}{} $ ist eine $ \bar{\sigma} $-Struktur mit $ \m{M}_i \subseteq \m{M} $ für alle $ i \in \alpha $.
    $ \m{M} $ kann sich wesentlich von den $ \m{M}_i $'s unterscheiden:
    \begin{bsp}
        $ \alpha = (\mathbb{N}, \leq) $, $ \m{M}_i = (\{1, ..., i\}, \leq) $ ($ i \in \alpha $).
        Dann $ \m{M} = (\mathbb{N}, \leq) $.
        \begin{itemize}
            \item $ \varphi = \exists y \forall x: x \leq y $
            \item $ \m{M}_i \models \varphi $ für alle $ i \in \alpha $
            \item Aber $ \m{M} \not \models \varphi $
        \end{itemize}
    \end{bsp}

    Eine Kette von $ \bar{\sigma} $-Strukturen heißt \underline{elementare Kette}, falls $ \forall i, j \in \alpha: i < j \Rightarrow \m{M}_i \preceq \m{M}_j $.

    \begin{satz}
        Sei $ (\m{M}_i)_{i \in \alpha} $ eine elementare Kette von $ \bar{\sigma} $-Strukturen und $ \m{M} = \bigcup_{i \in \alpha} \m{M}_i $.
        Dann gilt
        \begin{equation*}
            \forall i \in \alpha: \m{M}_i \preceq \m{M}
        \end{equation*}
    \end{satz}

    \begin{proof}
        Wir zeigen
        \begin{equation}
        \label{eq:elem-kette-zz}
            \begin{aligned}
                \forall \varphi = \varphi(v_0, ..., v_n) \forall i \in \alpha \forall a_0, ..., a_n \in M_i: \\ \m{M}_i \models \varphi[a_0, ..., a_n] \Leftrightarrow \m{M} \models \varphi[a_0, ..., a_n]
            \end{aligned}
        \end{equation}

        Beweis von \eqref{eq:elem-kette-zz} über Induktion über Formelaufbau.
        \begin{description}
            \item[IA] Sei $ \varphi $ eine Atomformel: Klar, da $ \m{M}_i \subseteq \m{M} $.
            \item[IS] Es gelte \eqref{eq:elem-kette-zz} für $ \varphi $ und $ \psi $.
            z.Z. \eqref{eq:elem-kette-zz} gilt für $ \neg \varphi $, $ \varphi \land \psi $, $ (\exists v_n) \varphi $.
            Für $ \neg $ und $ \land $ klar.

            Sei $ \chi(v_0, ..., v_{n -1 }) = (\exists v_n) \varphi(v_0, ..., v_n) $.
            Seien $ a_0, ..., a_{n - 1} \in M_i $.
            \begin{align*}
                & \m{M}_i \models \chi[a_0, ..., a_{n - 1}] \\
                \Leftrightarrow & \exists a_n \in M_i: M_i \models \varphi[a_0, ..., a_{n - 1}, a_n] \\
                \Rightarrow & \m{M} \models \varphi[a_0, ..., a_{n - 1}, a_n] & \text{Voraussetzung} \\
                \Rightarrow & \m{M} \models \chi[a_0, ..., a_{n - 1}]
            \end{align*}

            \begin{align*}
                & \m{M} \models \chi[a_0, ..., a_{n - 1}] \\
                \Rightarrow & \exists a_n \in M: \m{M} \models \varphi[a_0, ..., a_{n - 1}, a_n] \\
                \Rightarrow & \exists j \in \alpha: a_n \in M_j
            \end{align*}
            Fallunterscheidung:
            \begin{description}
                \item[$ j \leq i $]
                \begin{align*}
                    \Rightarrow & a_n \in M_i \\
                    \Rightarrow & \m{M}_i \models \varphi[a_0, ..., a_{n - 1}, a_n] & \text{Voraussetzung} \\
                    \Rightarrow & \m{M}_i \models \chi[a_0, ..., a_{n - 1}]
                \end{align*}
                \item[$ i < j $]
                \begin{align*}
                    \Rightarrow & a_0, ..., a_{k - 1} \in \m{M}_j \\
                    \Rightarrow & \m{M}_j \models \varphi[a_0, ..., a_{n - 1}, a_n] & \text{Voraussetzung} \\
                    \Rightarrow & \m{M}_j \models \chi[a_0, ..., a_{n - 1}] \\
                    \Rightarrow & \m{M}_i \models \chi[a_0, ..., a_{n -1}] & \m{M}_i \preceq \m{M}_j
                \end{align*}
            \end{description}
        \end{description}
    \end{proof}

    \begin{proof}[Beweis zu Satz \ref{satz:kompaktheitssatz}]
        ~\par
        \begin{description}
            \item[``$ \Rightarrow $''] Trivial.
            \item[``$ \Leftarrow $'']
        \end{description}
    \end{proof}

\end{document}
